%%%%%%%%%%%%%%%% Document settings %%%%%%%%%%%%%%%%%%%%
\NeedsTeXFormat{LaTeX2e}

\RequirePackage{ifpdf}	%% detection of pdfTeX or DVI
\ifpdf
	\documentclass[a4paper,12pt]{article}
\else
	\documentclass[a4paper,12pt,dvipdfmx]{article}
\fi

\usepackage{graphicx}                %% special graphics
\usepackage{color}                   %% allow colour
\usepackage{latexsym}                %% symbols package
\usepackage{amsmath}                 %% special mathematical functions
\usepackage{xfrac}
\usepackage{siunitx}
\usepackage{theorem}
\usepackage{amsfonts}
\usepackage{color}
\usepackage{tikz}
%\usepackage{multirow}                %% multirow tables



 %%############### Choose LANGUAGE
 %%% English
%	\usepackage[english]{babel}
%	\selectlanguage{english}

 %%% German
	\usepackage[english, german]{babel}
	\selectlanguage{german}
   % %% In some cases you might need that
   % \usepackage{german}
	\usepackage[latin1]{inputenc}
	\usepackage[T1]{fontenc}
	\usepackage{textcomp}


 %%############### Functions and Hyphenation rules
\newcommand{\bbma} {\begin{bmatrix} }
\newcommand{\ebma} {\end{bmatrix}}
\newcommand{\real} {\mathbb{R}}
\newcommand{\vs}[1]{\vspace{#1mm}}


\hyphenation{Trenn-ungs-re-geln meh-re-re}


 %%############### Enumeration of figures and equations
\numberwithin{figure}{section}
\numberwithin{equation}{section}
\newtheorem{definition}{Definition}[section]
\newtheorem{theorem}{Theorem}[section]


 %%############### Figures
\graphicspath{{figures/}} 				% Folder with the figures

\ifpdf % pdfTeX
	\DeclareGraphicsExtensions{.pdf,.png,.jpg,.gif,.pdftex}
	\DeclareGraphicsRule{.pdftex}{pdf}{*}{}	% for xfig exported files
	\newcommand{\xfig}[2]{\scalebox{#1}{\input{"figures/#2.pdftex_t"}}} % XFIG

\else % DVI
	\DeclareGraphicsExtensions{.eps,.pstex}
	\DeclareGraphicsRule{.pstex}{eps}{*}{}		% for xfig exported files
	\newcommand{\xfig}[2]{\scalebox{#1}{\input{"figures/#2.pstex_t"}}} % XFIG
	%\usepackage{psfrag}		% might be useful
\fi

 % Other figures (e.g., MATLAB  EPS or PDF figures)
\newcommand{\afig}[2]{\includegraphics[scale=#1]{#2}}

 % figures defined in TeX files
\newcommand{\tfig}[2]{\scalebox{#1}{\input{"#2.tex"}}}







 %%############### Hyper References Settings
%\ifpdf % pdfTeX
%\else
\usepackage[
%a4paper=true;
%letterpaper=false;
hypertexnames = false,             % hyperlinks
bookmarks=true,
bookmarksnumbered=true,%
linkbordercolor={0 1 1},%        % cyan
urlbordercolor={0 0 1},%         % magenta
plainpages=false,                % false - to fix the hyperpage links for index
pdfborder={0 0 0},
pdfsubject={Control Systems}]
%pdfborderstyle={/S/U/W 1}]%
{hyperref}
%\fi





 %%############### Page Size Settings
%%% WAIT - before changing the settings make a back-up copy!!!
%%% For more information about the meaning of the settings check
%%% "The not so short introduction to LaTeX", page 125

\setlength{\hoffset}{-1in}%               % To eliminate the standard offset
\setlength{\voffset}{-1in}%
\setlength{\evensidemargin}{23mm}%
\setlength{\oddsidemargin}{35mm}%
\setlength{\topmargin}{10mm}%
\setlength{\headheight}{30pt}%

\setlength{\headsep}{10mm}%
\setlength{\textheight}{230mm}%
\setlength{\textwidth}{160mm}%
\setlength{\marginparsep}{0mm}%
\setlength{\marginparwidth}{10mm}%
\setlength{\footskip}{10mm}%

\renewcommand{\textfraction}{0}%
\renewcommand{\topfraction}{0.9}%
\renewcommand{\bottomfraction}{0.9}%
\renewcommand{\floatpagefraction}{1.5}

\widowpenalty=10000      % penalty for creating widow line at top of page




 %%############### Paragraph settings

\setlength{\parindent}{0pt}%      % Paragraphs are not indent
\setlength{\parskip}{5pt}%        % space between paragraphs

\setlength{\labelwidth}{0.5cm}%
\setlength{\labelsep}{0.5cm}%

\renewcommand{\baselinestretch}{1.0}




 %%############### % Numbered items
%\setcounter{secnumdepth}{2}            % Number until 3th depth X.Y.Z
%
%\numberwithin{figure}  {section}       % Numbering of for the figures
%\numberwithin{equation}{section}
%
%\newtheorem{definition} {Definition}[section]
%\newtheorem{theorem}    {Theorem}   [section]




 %%###############  Page Style settings
%\usepackage{fancyhdr}                %% extended page layout formating

%\pagestyle{fancy}
%
%\renewcommand{\chaptermark}[1]{\markboth{\thechapter. \MakeUppercase{#1}}{}}
%\renewcommand{\chaptermark}[1]{\markboth{\MakeUppercase{#1}}{}}
%\fancyhf{}%
%\renewcommand{\headrulewidth}{0.4pt}
%\renewcommand{\footrulewidth}{0.0pt}
%
% %% Single sinded
%    \fancyhead[R]{\thepage}%%
%    \fancyhead[L]{\rightmark}%%
% \fancypagestyle{plain}{%
%    \fancyhf{} % get rid of headers
%    \renewcommand{\headrulewidth}{0pt}% get rid of the lines
%    \renewcommand{\footrulewidth}{0pt}% get rid of the lines
%}%
%\makeatletter
%\def\cleardoublepage{\clearpage\if@twoside \ifodd\c@page%
%\else \thispagestyle{empty} \hbox{} \newpage%    
%\fi\fi}%
%\makeatother


