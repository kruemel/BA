\section*{Einleitung}

Sie stehen am Anfang Ihrer Studien- oder Diplomarbeit bzw.\ Sie
haben bereits Ergebnisse in Form von Experimenten, Algorithmen,
Methoden, handschriftlichen Notizen etc.\ und fragen sich nun, wie
Sie das alles zu Papier bringen sollen?

Dieser Text ist sowohl eine Anleitung zur Niederschrift Ihrer
Studien- oder Diplomarbeit als auch eine LaTeX-Vorlage, die als
Vorlage zum Aufbau Ihrer eigenen Arbeit dienen kann.

Nachdem Sie sich den Text durchgelesen haben und die LaTeX-Dateien
angesehen haben, die diesen Text erzeugt haben, sollten Sie in der
Lage sein, Ihre eigene Arbeit in wissenschaftlich korrekter,
anschaulicher und ansprechender �u�erer Form zu produzieren.

Latex hat sich in den letzten Jahren stetig weiterentwickelt, 
so dass einige Befehle, die in �lteren tex-Dateien benutzt wurden
heute immer noch zu einem fehlerfreien Dokument f�hren, allerdings gibt
es meist bessere Ersetzungen. Eine gute Zusammenfassung f�r deutsche 
Dokumente finden Sie bei
\href{ftp://ftp.dante.de/pub/tex/info/german/l2tabu/l2tabu.pdf}{dante.de}.

Die Arbeit gliedert sich in drei Abschnitte. Im Kapitel
\ref{a:latex} wird anhand von Beispielen die Erstellung eines
LaTeX-Dokumentes vorgef�hrt. Das Kapitel \ref{a:stil} werden die
wichtige Punkte zum wissenschaftlichen Schreibstil erl�utert. Das
Kapitel \ref{a:ab_rts} f�hrt Sie dann noch in die Besonderheiten des
Arbeitsbereichs Regelungstechnik ein und stellt die Ihnen zur
Verf�gung stehenden Werkzeuge vor.
