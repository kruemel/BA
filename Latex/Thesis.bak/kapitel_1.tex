\section{Das Schreiben mit LaTeX}\label{a:latex}

\subsection{Das Besondere an LaTeX}

In diesem Kapitel werden Sie einige n�tzliche Hinweise zur
Erstellung von Studien- und Diplomarbeiten mit LaTeX erhalten.
Wichtig zu wissen ist, dass LaTeX kein WYSIWYG (what you see is
what you get) Programm ist wie andere Texteditoren wie z.B.
Microsoft Word. Es ist vielmehr eine Art Programmiersprache, mit
der Sie den logischen Zusammenhang Ihres Textes durch geeignete
Befehle zun�chst beschreiben. Dieser ''Quellcode'' ist die sog.\
LaTeX-Datei.

Es gibt wie bei Programmen eine compilierbare Datei f�r das
Hauptprogramm und u.U.\ weitere Dateien f�r Unterprogramme, die
eingebunden werden (sinnvollerweise sind das hier die Kapitel).
Der Compiler erzeugt aus dem Quellcode entweder eine sog.\
Dvi-Datei (f�r ''device independent'') (LaTeX) oder direkt eine
Pdf-Datei (PdfLaTeX). Aus der Dvi-Datei k�nnen Sie mit dem
Programm dvi2ps dann eine Postscript-Datei erstellen, die dann auf
geeigneten Druckern ausgedruckt werden kann.

Dies gibt Ihnen z.B.\ auch die M�glichkeit im Quellcode Kommentare
einzuf�gen, die im endg�ltigen Text nicht sichtbar sind. Es kommt
dabei auch nicht auf die Zeilenumbr�che im Quellcode an, da diese
erst durch den Compilierungsvorgang erzeugt werden. Das Layout der
fertigen Arbeit k�nnen Sie getrost am letzten Tag Ihrer Arbeit
machen; wenn alle logischen Verkn�pfungen und der Text als solcher
sowie die Bilder stimmig sind, dauert das ca.\ einen Tag. Bedenken
Sie, dass auch das Ausdrucken, insbesondere bei farbigen Seiten
Zeit in Anspruch nimmt!

Die Vorteile von LaTeX werden insbesondere bei Umstellungen des
Textes und beim Formelsatz deutlich. Durch die logische
Referenzierung stimmen die Referenzen immer, die Druckqualit�t von
Formeln wird von anderen Programmen nicht erreicht. Die
Einfachheit des Zitierens ist gerade f�r einen Anf�nger sehr
hilfreich.

Der folgende Text als solcher ist nicht sonderlich sinnvoll, wenn
Sie aber gleichzeitig die Datei mit dem Quellcode lesen, kann man
sehr gut verstehen, wie verschiedene Elemente erzeugt werden. Sie
sollten dann diese Datei auch selbst compilieren und das Ergebnis
vergleichen. Wenn es nicht identisch ist, ist u.U.\ die
LaTeX-Konfiguration nicht korrekt und muss dann korrigiert werden.

\subsection{Normaler Text}

Normalen Text zu schreiben ist ganz einfach. Man nutzt einen
beliebigen Editor und schreibt einfach den Text, ohne auf die
Formatierung, Zeilenumbr�che etc. zu achten. Will man einen neuen
Abschnitt beginnen, so l�sst man einfach eine oder mehrere
Leerzeilen...

Hier sind wir nun im neuen Abschnitt. Die Schriftgr��e wird
�brigens im Kopfteil des Hauptdokumentes definiert. Man kann
Hervorhebungen durch andere Schriftarten machen. Normalerweise
werden neu definierte Begriffe in \textit{italic}, wichtige
Aussagen in \textbf{bold} und Programmzeilen mit \verb"\verb"
gesetzt.

\subsubsection{Gliederungsebenen}

Man kann feinere Unterteilungen der Abschnitte durch den Befehl
''subsubsection'' erreichen. Diese Ebene sollte die unterste Ihrer
Arbeit sein.

\subsubsection*{Gleiche Ebene, aber nicht numeriert}

Sehen Sie in den Quellcode, um zu wissen, wie das erreicht wird!

\subsection{Besondere Objekte}


\subsubsection{Bilder}\label{b:picturesubsection}

Bild \ref{f:picfirstfigure} zeigt ein Blockschaltbild.

\begin{figure}[ht]		% h - here, t - top, b - bottom, p - page, ! - try hard
  \centering
  \afig{1}{example}			% {scaling}{Figure from MATLAB, picture, etc.}
  \caption{Das erste Bild}
  \label{f:picfirstfigure}
\end{figure}

Wenn das Datei durch DVI kompiliert sind (latex $\rightarrow$ dvi $\rightarrow$ pdf, oder latex $\rightarrow$ dvi $\rightarrow$ ps $\rightarrow$ pdf) kann Bild \ref{f:picfirstfigure} PS oder EPS sein. Wenn aber, man kompiliert direkt mit pdfTeX (latex $\rightarrow$ pdf), die Figuren m�ssen JPEG, PDF, PNG, usw. sein -- kein PS oder EPS.


Bild \ref{f:xfigfigure} ist ein Beispiel f�r Diagramm, das mit XFig erzeugt ist. 

\begin{figure}[ht]
  \centering
  \xfig{1}{carts}			% {scaling}{XFig figure}
  \caption{Das zweite Bild}
  \label{f:xfigfigure}
\end{figure}







\subsubsection{Formeln}

Formeln wie diese
\begin{eqnarray}
\ddot{\phi}_1
    &=&
    \frac{M_1+l_1 \sin \phi_2
    (m_2 l_{s2} + m_3 l_{s3})
    (\dot{\phi}_2^2+2 \dot{\phi}_1\dot{\phi}_2 )
    -f_1 \dot{\phi}_1}
    {\theta_1+\theta_2+\theta_3+2l_1 (m_1 l_{s2}+m_3 l_{s3} \cos \phi_2)+
    m_3 (l_1^2+l_2^2)+m_l l_1^2}  \,,  \label{e:eqnfirst} \\
\ddot{\phi}_2
    &=&
    \frac{M_2 + l_1 \sin \phi_2 (m_2 l_{s2} + m_3 l_{s3})
    \dot{\phi}_1^2 +2 - \phi_2 \dot{\phi}_2}
    {\theta_1+\theta_2+\theta_3} \, \label{e:eqnsecond}
\end{eqnarray}
k�nnen in sch�nem Satz ausgedruckt werden. Bitte beachten Sie,
dass auch Formeln zu den S�tzen geh�ren und ebenso Satzzeichen
enthalten k�nnen!


\subsubsection{Symbole}

Symbole wie $\Omega$ k�nnen auch einfach in den Fliesstext mit
aufgenommen werden. S�tze fangen {\bf nie} mit einem Symbol an!


\subsubsection{Zitate}

Zitate werden einfach mit Komma an den Satz angeh�ngt,
\cite{Lun1}. Nachdem man das Programm BiBTeX aufgerufen hat, wird
das Literaturverzeichnis automatisch erstellt. Die Auswahl eines
Zitierstiles erfolgt in der Haupt-Datei.


\subsubsection{Inhaltsverzeichnis}

Das Inhaltsverzeichnis wird automatisch durch LaTeX erstellt. Dazu
schreibt das Programm bei jedem Compilierungslauf  sog.\
aux-Dateien (auxiliary), die alle f�r das Inhaltsverzeichnis
wichtigen Elemente enthalten. Diese Dateien werden dann beim
n�chsten Compilieren mit eingebunden. Wenn


\subsubsection{Referenzen}\label{b:subsecrefer}

Man kann auf Bilder wie das Bild \ref{f:picfirstfigure}, Gleichungen
(\ref{e:eqnsecond}), oder ganze Abschnitte \ref{b:picturesubsection}
verweisen. Sehen Sie, wie das geht? Hierin liegt die eigentliche
St�rke des Programms!
